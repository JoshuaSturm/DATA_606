\documentclass[]{article}
\usepackage{lmodern}
\usepackage{amssymb,amsmath}
\usepackage{ifxetex,ifluatex}
\usepackage{fixltx2e} % provides \textsubscript
\ifnum 0\ifxetex 1\fi\ifluatex 1\fi=0 % if pdftex
  \usepackage[T1]{fontenc}
  \usepackage[utf8]{inputenc}
\else % if luatex or xelatex
  \ifxetex
    \usepackage{mathspec}
  \else
    \usepackage{fontspec}
  \fi
  \defaultfontfeatures{Ligatures=TeX,Scale=MatchLowercase}
\fi
% use upquote if available, for straight quotes in verbatim environments
\IfFileExists{upquote.sty}{\usepackage{upquote}}{}
% use microtype if available
\IfFileExists{microtype.sty}{%
\usepackage{microtype}
\UseMicrotypeSet[protrusion]{basicmath} % disable protrusion for tt fonts
}{}
\usepackage[margin=1in]{geometry}
\usepackage{hyperref}
\hypersetup{unicode=true,
            pdftitle={DATA 606 - Homework 4},
            pdfauthor={Joshua Sturm},
            pdfborder={0 0 0},
            breaklinks=true}
\urlstyle{same}  % don't use monospace font for urls
\usepackage{color}
\usepackage{fancyvrb}
\newcommand{\VerbBar}{|}
\newcommand{\VERB}{\Verb[commandchars=\\\{\}]}
\DefineVerbatimEnvironment{Highlighting}{Verbatim}{commandchars=\\\{\}}
% Add ',fontsize=\small' for more characters per line
\usepackage{framed}
\definecolor{shadecolor}{RGB}{248,248,248}
\newenvironment{Shaded}{\begin{snugshade}}{\end{snugshade}}
\newcommand{\KeywordTok}[1]{\textcolor[rgb]{0.13,0.29,0.53}{\textbf{#1}}}
\newcommand{\DataTypeTok}[1]{\textcolor[rgb]{0.13,0.29,0.53}{#1}}
\newcommand{\DecValTok}[1]{\textcolor[rgb]{0.00,0.00,0.81}{#1}}
\newcommand{\BaseNTok}[1]{\textcolor[rgb]{0.00,0.00,0.81}{#1}}
\newcommand{\FloatTok}[1]{\textcolor[rgb]{0.00,0.00,0.81}{#1}}
\newcommand{\ConstantTok}[1]{\textcolor[rgb]{0.00,0.00,0.00}{#1}}
\newcommand{\CharTok}[1]{\textcolor[rgb]{0.31,0.60,0.02}{#1}}
\newcommand{\SpecialCharTok}[1]{\textcolor[rgb]{0.00,0.00,0.00}{#1}}
\newcommand{\StringTok}[1]{\textcolor[rgb]{0.31,0.60,0.02}{#1}}
\newcommand{\VerbatimStringTok}[1]{\textcolor[rgb]{0.31,0.60,0.02}{#1}}
\newcommand{\SpecialStringTok}[1]{\textcolor[rgb]{0.31,0.60,0.02}{#1}}
\newcommand{\ImportTok}[1]{#1}
\newcommand{\CommentTok}[1]{\textcolor[rgb]{0.56,0.35,0.01}{\textit{#1}}}
\newcommand{\DocumentationTok}[1]{\textcolor[rgb]{0.56,0.35,0.01}{\textbf{\textit{#1}}}}
\newcommand{\AnnotationTok}[1]{\textcolor[rgb]{0.56,0.35,0.01}{\textbf{\textit{#1}}}}
\newcommand{\CommentVarTok}[1]{\textcolor[rgb]{0.56,0.35,0.01}{\textbf{\textit{#1}}}}
\newcommand{\OtherTok}[1]{\textcolor[rgb]{0.56,0.35,0.01}{#1}}
\newcommand{\FunctionTok}[1]{\textcolor[rgb]{0.00,0.00,0.00}{#1}}
\newcommand{\VariableTok}[1]{\textcolor[rgb]{0.00,0.00,0.00}{#1}}
\newcommand{\ControlFlowTok}[1]{\textcolor[rgb]{0.13,0.29,0.53}{\textbf{#1}}}
\newcommand{\OperatorTok}[1]{\textcolor[rgb]{0.81,0.36,0.00}{\textbf{#1}}}
\newcommand{\BuiltInTok}[1]{#1}
\newcommand{\ExtensionTok}[1]{#1}
\newcommand{\PreprocessorTok}[1]{\textcolor[rgb]{0.56,0.35,0.01}{\textit{#1}}}
\newcommand{\AttributeTok}[1]{\textcolor[rgb]{0.77,0.63,0.00}{#1}}
\newcommand{\RegionMarkerTok}[1]{#1}
\newcommand{\InformationTok}[1]{\textcolor[rgb]{0.56,0.35,0.01}{\textbf{\textit{#1}}}}
\newcommand{\WarningTok}[1]{\textcolor[rgb]{0.56,0.35,0.01}{\textbf{\textit{#1}}}}
\newcommand{\AlertTok}[1]{\textcolor[rgb]{0.94,0.16,0.16}{#1}}
\newcommand{\ErrorTok}[1]{\textcolor[rgb]{0.64,0.00,0.00}{\textbf{#1}}}
\newcommand{\NormalTok}[1]{#1}
\usepackage{graphicx,grffile}
\makeatletter
\def\maxwidth{\ifdim\Gin@nat@width>\linewidth\linewidth\else\Gin@nat@width\fi}
\def\maxheight{\ifdim\Gin@nat@height>\textheight\textheight\else\Gin@nat@height\fi}
\makeatother
% Scale images if necessary, so that they will not overflow the page
% margins by default, and it is still possible to overwrite the defaults
% using explicit options in \includegraphics[width, height, ...]{}
\setkeys{Gin}{width=\maxwidth,height=\maxheight,keepaspectratio}
\IfFileExists{parskip.sty}{%
\usepackage{parskip}
}{% else
\setlength{\parindent}{0pt}
\setlength{\parskip}{6pt plus 2pt minus 1pt}
}
\setlength{\emergencystretch}{3em}  % prevent overfull lines
\providecommand{\tightlist}{%
  \setlength{\itemsep}{0pt}\setlength{\parskip}{0pt}}
\setcounter{secnumdepth}{0}
% Redefines (sub)paragraphs to behave more like sections
\ifx\paragraph\undefined\else
\let\oldparagraph\paragraph
\renewcommand{\paragraph}[1]{\oldparagraph{#1}\mbox{}}
\fi
\ifx\subparagraph\undefined\else
\let\oldsubparagraph\subparagraph
\renewcommand{\subparagraph}[1]{\oldsubparagraph{#1}\mbox{}}
\fi

%%% Use protect on footnotes to avoid problems with footnotes in titles
\let\rmarkdownfootnote\footnote%
\def\footnote{\protect\rmarkdownfootnote}

%%% Change title format to be more compact
\usepackage{titling}

% Create subtitle command for use in maketitle
\newcommand{\subtitle}[1]{
  \posttitle{
    \begin{center}\large#1\end{center}
    }
}

\setlength{\droptitle}{-2em}
  \title{DATA 606 - Homework 4}
  \pretitle{\vspace{\droptitle}\centering\huge}
  \posttitle{\par}
  \author{Joshua Sturm}
  \preauthor{\centering\large\emph}
  \postauthor{\par}
  \predate{\centering\large\emph}
  \postdate{\par}
  \date{10/15/2017}


\begin{document}
\maketitle

{
\setcounter{tocdepth}{2}
\tableofcontents
}
\subsection{4.4 Heights of adults}\label{heights-of-adults}

Researchers studying anthropometry collected body girth measurements and
skeletal diameter measurements, as well as age, weight, height and
gender, for 507 physically active individuals. The histogram below shows
the sample distribution of heights in centimeters.

\subsubsection{(a)}\label{a}

What is the point estimate for the average height of active individuals?
What about the median?

\paragraph{Solution}\label{solution}

The point estimate for the average is the mean: 171.1. Similarly, the
point estimate for the median would be 170.3.

\subsubsection{(b)}\label{b}

What is the point estimate for the standard deviation of the heights of
active individuals? What about the IQR?

\paragraph{Solution}\label{solution-1}

The point estimate for the standard deviation is 9.4. The IQR is Q3 -
Q1: 177.8 - 163.8 = 14.

\subsubsection{(c)}\label{c}

Is a person who is 1m 80cm (180 cm) tall considered unusually tall? And
is a person who is 1m 55cm (155cm) considered unusually short? Explain
your reasoning.

\paragraph{Solution}\label{solution-2}

They are both unusual in that they are above the third quartile and
below the first quartile, respectively.

\begin{Shaded}
\begin{Highlighting}[]
\NormalTok{n <-}\StringTok{ }\DecValTok{507}
\NormalTok{mean <-}\StringTok{ }\FloatTok{171.1}
\NormalTok{sd <-}\StringTok{ }\FloatTok{9.4}
\NormalTok{shorter <-}\StringTok{ }\DecValTok{155}
\NormalTok{taller <-}\StringTok{ }\DecValTok{180}
\CommentTok{# Calculate z-scores:}
\NormalTok{z_shorter <-}\StringTok{ }\NormalTok{(shorter }\OperatorTok{-}\StringTok{ }\NormalTok{mean) }\OperatorTok{/}\StringTok{ }\NormalTok{sd}
\NormalTok{z_taller <-}\StringTok{ }\NormalTok{(taller }\OperatorTok{-}\StringTok{ }\NormalTok{mean) }\OperatorTok{/}\StringTok{ }\NormalTok{sd}
\NormalTok{z_shorter}
\end{Highlighting}
\end{Shaded}

\begin{verbatim}
## [1] -1.712766
\end{verbatim}

\begin{Shaded}
\begin{Highlighting}[]
\NormalTok{z_taller}
\end{Highlighting}
\end{Shaded}

\begin{verbatim}
## [1] 0.9468085
\end{verbatim}

Looking at the z-scores, though, the person with a height of 155cm is
more unusual, since they are considerably farther from the mean.

\subsubsection{(d)}\label{d}

The researchers take another random sample of physically active
individuals. Would you expect the mean and the standard deviation of
this new sample to be the ones given above? Explain your reasoning.

\paragraph{Solution}\label{solution-3}

I would expect the mean and sd to be similar but different from the
first sample.

\subsubsection{(e)}\label{e}

The sample means obtained are point estimates for the mean height of all
active individuals, if the sample of individuals is equivalent to a
simple random sample. What measure do we use to quantify the variability
of such an estimate (Hint: recall that
\(SD_{\bar x} = \frac{\sigma}{\sqrt{n}}\))? Compute this quantity using
the data from the original sample under the condition that the data are
a simple random sample.

\paragraph{Solution}\label{solution-4}

We can use the Standard Error: \(SE = \frac{\sigma}{\sqrt{n}}\).

\begin{Shaded}
\begin{Highlighting}[]
\NormalTok{SE <-}\StringTok{ }\NormalTok{sd }\OperatorTok{/}\StringTok{ }\KeywordTok{sqrt}\NormalTok{(n)}
\NormalTok{SE}
\end{Highlighting}
\end{Shaded}

\begin{verbatim}
## [1] 0.4174687
\end{verbatim}

\subsection{4.14 Thanksgiving spending, Part
I.}\label{thanksgiving-spending-part-i.}

The 2009 holiday retail season, which kicked off on November 27, 2009
(the day after Thanksgiving), had been marked by somewhat lower
self-reported consumer spending than was seen during the comparable
period in 2008. To get an estimate of consumer spending, 436 randomly
sampled American adults were surveyed. Daily consumer spending for the
six-day period after Thanksgiving, spanning the Black Friday weekend and
Cyber Monday, averaged \$84.71. A 95\% confidence interval based on this
sample is (\$80.31, \$89.11). Determine whether the following statements
are true or false, and explain your reasoning.

\subsubsection{(a)}\label{a-1}

We are 95\% confident that the average spending of these 436 American
adults is between \$80.31 and \$89.11.

\paragraph{Solution}\label{solution-5}

\textbf{False.} The average spent in this sample is \$84.71. We are 95\%
confident that the average for the entire \emph{population} is between
\$80.31 and \$89.11.

\subsubsection{(b)}\label{b-1}

This confidence interval is not valid since the distribution of spending
in the sample is right skewed.

\paragraph{Solution}\label{solution-6}

\textbf{False.} So long as the sample is larger than 30, it's valid, and
the skew should not matter.

\subsubsection{(c)}\label{c-1}

95\% of random samples have a sample mean between \$80.31 and \$89.11.

\paragraph{Solution}\label{solution-7}

\textbf{False.} This confidence interval provides us with an idea of
where the population mean is. A sample can differ from this, since the
confidence interval is based on the mean and sd, which varies between
samples.

\subsubsection{(d)}\label{d-1}

We are 95\% confident that the average spending of all American adults
is between \$80.31 and \$89.11.

\paragraph{Solution}\label{solution-8}

\textbf{True.} This is the idea behind the confidence interval - to find
the population mean.

\subsubsection{(e)}\label{e-1}

A 90\% confidence interval would be narrower than the 95\% confidence
interval since we don't need to be as sure about our estimate.

\paragraph{Solution}\label{solution-9}

\textbf{True.} If we're not measuring as precisely, the interval will be
smaller, and less accurate.

\subsubsection{(f)}\label{f}

In order to decrease the margin of error of a 95\% confidence interval
to a third of what it is now, we would need to use a sample 3 times
larger.

\paragraph{Solution}\label{solution-10}

\textbf{False.} The formula for the margin of error is:
\(MOE = z* \times SE\). If we want \(\frac{1}{3}\times SE\), we get
\(\frac{\sigma}{\sqrt(3^2 * n)}\), so we'd have to use a sample 9 times
larger.

\subsubsection{(g)}\label{g}

The margin of error is 4.4.

\paragraph{Solution}\label{solution-11}

\textbf{True.}

\begin{Shaded}
\begin{Highlighting}[]
\NormalTok{confl <-}\StringTok{ }\FloatTok{80.31}
\NormalTok{confu <-}\StringTok{ }\FloatTok{89.11}
\NormalTok{moe <-}\StringTok{ }\NormalTok{(confu }\OperatorTok{-}\StringTok{ }\NormalTok{confl) }\OperatorTok{/}\StringTok{ }\DecValTok{2}
\NormalTok{moe}
\end{Highlighting}
\end{Shaded}

\begin{verbatim}
## [1] 4.4
\end{verbatim}

\subsection{4.24 Gifted children, Part
I.}\label{gifted-children-part-i.}

Researchers investigating characteristics of gifted children collected
data from schools in a large city on a random sample of thirty-six
children who were identified as gifted children soon after they reached
the age of four. The following histogram shows the distribution of the
ages (in months) at which these children first counted to 10
successfully. Also provided are some sample statistics.

\subsubsection{(a)}\label{a-2}

Are conditions for inference satisfied?

\paragraph{Solution}\label{solution-12}

I believe so. Since the sample is from a large pool (school children in
a large city), we can assume they're independent. Furthermore, we have a
sample size of 36, so \(n = 36 \geq 30\). Lastly, there doesn't appear
to be any strong skew in the histogram plot.

\subsubsection{(b)}\label{b-2}

Suppose you read online that children first count to 10 successfully
when they are 32 months old, on average. Perform a hypothesis test to
evaluate if these data provide convincing evidence that the average age
at which gifted children fist count to 10 successfully is less than the
general average of 32 months. Use a significance level of 0.10.

\paragraph{Solution}\label{solution-13}

Null hypothesis \(H_0: \mu = 32\) months.\\
Alternate hypothesis \(H_A: \mu < 32\) months.\\
\(\alpha = 0.10\).\\
Since we're only interested if it's less than the average, it's a
one-sided hypothesis test.

\begin{Shaded}
\begin{Highlighting}[]
\NormalTok{n <-}\StringTok{ }\DecValTok{36}
\NormalTok{mean <-}\StringTok{ }\FloatTok{30.69}
\NormalTok{sd <-}\StringTok{ }\FloatTok{4.31}
\NormalTok{SE <-}\StringTok{ }\NormalTok{sd }\OperatorTok{/}\StringTok{ }\KeywordTok{sqrt}\NormalTok{(n)}
\NormalTok{z <-}\StringTok{ }\NormalTok{((mean }\OperatorTok{-}\StringTok{ }\DecValTok{32}\NormalTok{)}\OperatorTok{/}\NormalTok{SE)}
\end{Highlighting}
\end{Shaded}


\end{document}
