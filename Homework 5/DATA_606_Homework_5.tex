\documentclass[]{article}
\usepackage{lmodern}
\usepackage{amssymb,amsmath}
\usepackage{ifxetex,ifluatex}
\usepackage{fixltx2e} % provides \textsubscript
\ifnum 0\ifxetex 1\fi\ifluatex 1\fi=0 % if pdftex
  \usepackage[T1]{fontenc}
  \usepackage[utf8]{inputenc}
\else % if luatex or xelatex
  \ifxetex
    \usepackage{mathspec}
  \else
    \usepackage{fontspec}
  \fi
  \defaultfontfeatures{Ligatures=TeX,Scale=MatchLowercase}
\fi
% use upquote if available, for straight quotes in verbatim environments
\IfFileExists{upquote.sty}{\usepackage{upquote}}{}
% use microtype if available
\IfFileExists{microtype.sty}{%
\usepackage{microtype}
\UseMicrotypeSet[protrusion]{basicmath} % disable protrusion for tt fonts
}{}
\usepackage[margin=1in]{geometry}
\usepackage{hyperref}
\hypersetup{unicode=true,
            pdftitle={DATA 606 - Homework 5},
            pdfauthor={Joshua Sturm},
            pdfborder={0 0 0},
            breaklinks=true}
\urlstyle{same}  % don't use monospace font for urls
\usepackage{color}
\usepackage{fancyvrb}
\newcommand{\VerbBar}{|}
\newcommand{\VERB}{\Verb[commandchars=\\\{\}]}
\DefineVerbatimEnvironment{Highlighting}{Verbatim}{commandchars=\\\{\}}
% Add ',fontsize=\small' for more characters per line
\usepackage{framed}
\definecolor{shadecolor}{RGB}{248,248,248}
\newenvironment{Shaded}{\begin{snugshade}}{\end{snugshade}}
\newcommand{\KeywordTok}[1]{\textcolor[rgb]{0.13,0.29,0.53}{\textbf{#1}}}
\newcommand{\DataTypeTok}[1]{\textcolor[rgb]{0.13,0.29,0.53}{#1}}
\newcommand{\DecValTok}[1]{\textcolor[rgb]{0.00,0.00,0.81}{#1}}
\newcommand{\BaseNTok}[1]{\textcolor[rgb]{0.00,0.00,0.81}{#1}}
\newcommand{\FloatTok}[1]{\textcolor[rgb]{0.00,0.00,0.81}{#1}}
\newcommand{\ConstantTok}[1]{\textcolor[rgb]{0.00,0.00,0.00}{#1}}
\newcommand{\CharTok}[1]{\textcolor[rgb]{0.31,0.60,0.02}{#1}}
\newcommand{\SpecialCharTok}[1]{\textcolor[rgb]{0.00,0.00,0.00}{#1}}
\newcommand{\StringTok}[1]{\textcolor[rgb]{0.31,0.60,0.02}{#1}}
\newcommand{\VerbatimStringTok}[1]{\textcolor[rgb]{0.31,0.60,0.02}{#1}}
\newcommand{\SpecialStringTok}[1]{\textcolor[rgb]{0.31,0.60,0.02}{#1}}
\newcommand{\ImportTok}[1]{#1}
\newcommand{\CommentTok}[1]{\textcolor[rgb]{0.56,0.35,0.01}{\textit{#1}}}
\newcommand{\DocumentationTok}[1]{\textcolor[rgb]{0.56,0.35,0.01}{\textbf{\textit{#1}}}}
\newcommand{\AnnotationTok}[1]{\textcolor[rgb]{0.56,0.35,0.01}{\textbf{\textit{#1}}}}
\newcommand{\CommentVarTok}[1]{\textcolor[rgb]{0.56,0.35,0.01}{\textbf{\textit{#1}}}}
\newcommand{\OtherTok}[1]{\textcolor[rgb]{0.56,0.35,0.01}{#1}}
\newcommand{\FunctionTok}[1]{\textcolor[rgb]{0.00,0.00,0.00}{#1}}
\newcommand{\VariableTok}[1]{\textcolor[rgb]{0.00,0.00,0.00}{#1}}
\newcommand{\ControlFlowTok}[1]{\textcolor[rgb]{0.13,0.29,0.53}{\textbf{#1}}}
\newcommand{\OperatorTok}[1]{\textcolor[rgb]{0.81,0.36,0.00}{\textbf{#1}}}
\newcommand{\BuiltInTok}[1]{#1}
\newcommand{\ExtensionTok}[1]{#1}
\newcommand{\PreprocessorTok}[1]{\textcolor[rgb]{0.56,0.35,0.01}{\textit{#1}}}
\newcommand{\AttributeTok}[1]{\textcolor[rgb]{0.77,0.63,0.00}{#1}}
\newcommand{\RegionMarkerTok}[1]{#1}
\newcommand{\InformationTok}[1]{\textcolor[rgb]{0.56,0.35,0.01}{\textbf{\textit{#1}}}}
\newcommand{\WarningTok}[1]{\textcolor[rgb]{0.56,0.35,0.01}{\textbf{\textit{#1}}}}
\newcommand{\AlertTok}[1]{\textcolor[rgb]{0.94,0.16,0.16}{#1}}
\newcommand{\ErrorTok}[1]{\textcolor[rgb]{0.64,0.00,0.00}{\textbf{#1}}}
\newcommand{\NormalTok}[1]{#1}
\usepackage{graphicx,grffile}
\makeatletter
\def\maxwidth{\ifdim\Gin@nat@width>\linewidth\linewidth\else\Gin@nat@width\fi}
\def\maxheight{\ifdim\Gin@nat@height>\textheight\textheight\else\Gin@nat@height\fi}
\makeatother
% Scale images if necessary, so that they will not overflow the page
% margins by default, and it is still possible to overwrite the defaults
% using explicit options in \includegraphics[width, height, ...]{}
\setkeys{Gin}{width=\maxwidth,height=\maxheight,keepaspectratio}
\IfFileExists{parskip.sty}{%
\usepackage{parskip}
}{% else
\setlength{\parindent}{0pt}
\setlength{\parskip}{6pt plus 2pt minus 1pt}
}
\setlength{\emergencystretch}{3em}  % prevent overfull lines
\providecommand{\tightlist}{%
  \setlength{\itemsep}{0pt}\setlength{\parskip}{0pt}}
\setcounter{secnumdepth}{0}
% Redefines (sub)paragraphs to behave more like sections
\ifx\paragraph\undefined\else
\let\oldparagraph\paragraph
\renewcommand{\paragraph}[1]{\oldparagraph{#1}\mbox{}}
\fi
\ifx\subparagraph\undefined\else
\let\oldsubparagraph\subparagraph
\renewcommand{\subparagraph}[1]{\oldsubparagraph{#1}\mbox{}}
\fi

%%% Use protect on footnotes to avoid problems with footnotes in titles
\let\rmarkdownfootnote\footnote%
\def\footnote{\protect\rmarkdownfootnote}

%%% Change title format to be more compact
\usepackage{titling}

% Create subtitle command for use in maketitle
\newcommand{\subtitle}[1]{
  \posttitle{
    \begin{center}\large#1\end{center}
    }
}

\setlength{\droptitle}{-2em}
  \title{DATA 606 - Homework 5}
  \pretitle{\vspace{\droptitle}\centering\huge}
  \posttitle{\par}
  \author{Joshua Sturm}
  \preauthor{\centering\large\emph}
  \postauthor{\par}
  \predate{\centering\large\emph}
  \postdate{\par}
  \date{October 27, 2017}


\begin{document}
\maketitle

\subsection{5.6 Working backwards, Part
II.}\label{working-backwards-part-ii.}

A 90\% confidence interval for a population mean is (65,77). The
population distribution is approximately normal and the population
standard deviation is unknown. This confidence interval is based on a
simple random sample of 25 observations. Calculate the sample mean, the
margin of error, and the sample standard deviation.

\subsubsection{Solution}\label{solution}

Since the population distribution is approximately normal, the sample
means will be nearly normal even if \(n < 30\).\\
Formula for the sample mean: \(\frac{x_1 + x_2}{2}\).

\begin{Shaded}
\begin{Highlighting}[]
\NormalTok{n <-}\StringTok{ }\DecValTok{25}
\NormalTok{x1 <-}\StringTok{ }\DecValTok{65}
\NormalTok{x2 <-}\StringTok{ }\DecValTok{77}
\NormalTok{samp_mean <-}\StringTok{ }\NormalTok{(x1}\OperatorTok{+}\NormalTok{x2)}\OperatorTok{/}\DecValTok{2}
\end{Highlighting}
\end{Shaded}

Formula for the margin of error: \(\frac{x_2 - x_1}{2}\).

\begin{Shaded}
\begin{Highlighting}[]
\NormalTok{moe <-}\StringTok{ }\NormalTok{(x2}\OperatorTok{-}\NormalTok{x1)}\OperatorTok{/}\DecValTok{2}
\end{Highlighting}
\end{Shaded}

Formula for the standard deviation: \(SE = \frac{s}{\sqrt{n}}\).\\

\begin{Shaded}
\begin{Highlighting}[]
\NormalTok{df <-}\StringTok{ }\NormalTok{n}\OperatorTok{-}\DecValTok{1}
\NormalTok{c <-}\StringTok{ }\FloatTok{0.9}
\NormalTok{c2 <-}\StringTok{ }\NormalTok{c }\OperatorTok{+}\StringTok{ }\NormalTok{(}\DecValTok{1}\OperatorTok{-}\NormalTok{c)}\OperatorTok{/}\DecValTok{2}
\NormalTok{t24 <-}\StringTok{ }\KeywordTok{qt}\NormalTok{(c2, df)}
\end{Highlighting}
\end{Shaded}

\(ME = t_{24}^{*}SE \to SE = \frac{ME}{t_{24}^{*}} \to s = \frac{ME\sqrt{n}}{t_{24}^{*}} \to \frac{6\sqrt{25}}{t24}\)\\

\begin{Shaded}
\begin{Highlighting}[]
\NormalTok{s <-}\StringTok{ }\NormalTok{(moe}\OperatorTok{*}\KeywordTok{sqrt}\NormalTok{(}\DecValTok{25}\NormalTok{))}\OperatorTok{/}\NormalTok{(t24)}
\end{Highlighting}
\end{Shaded}

The standard deviation for the sample is 17.5348146.

\subsection{5.14 SAT scores.}\label{sat-scores.}

SAT scores of students at an Ivy League college are distributed with a
standard deviation of 250 points. Two statistics students, Raina and
Luke, want to estimate the average SAT score of students at this college
as part of a class project. They want their margin of error to be no
more than 25 points.\\
(a) Raina wants to use a 90\% confidence interval. How large a sample
should she collect?\\
(b) Luke wants to use a 99\% confidence interval. Without calculating
the actual sample size, determine whether his sample should be larger or
smaller than Raina's, and explain your reasoning.\\
(c) Calculate the minimum required sample size for Luke.

\subsubsection{(a)}\label{a}

\(ME = z*\frac{\sigma}{\sqrt{n}}, \quad ME = 25, \sigma = 250\).

\begin{Shaded}
\begin{Highlighting}[]
\CommentTok{# for 90% confidence interval, Z* = 1.645}
\NormalTok{z <-}\StringTok{ }\FloatTok{1.645}
\NormalTok{me <-}\StringTok{ }\DecValTok{25}
\NormalTok{sigma <-}\StringTok{ }\DecValTok{250}
\CommentTok{# 25 = 1.645*(250/sqrt(n))}
\NormalTok{n <-}\StringTok{ }\NormalTok{(z}\OperatorTok{*}\NormalTok{sigma}\OperatorTok{/}\NormalTok{me)}\OperatorTok{^}\DecValTok{2}
\end{Highlighting}
\end{Shaded}

Raina would need 270.6025 = 271 students.

\subsubsection{(b)}\label{b}

Since Luke wants a higher certainty, he'll need a higher z-value, which
will result in a larger sample size.

\subsubsection{(c)}\label{c}

\begin{Shaded}
\begin{Highlighting}[]
\CommentTok{# for 99% confidence interval, Z* = 2.58}
\NormalTok{z <-}\StringTok{ }\FloatTok{2.58}
\NormalTok{n <-}\StringTok{ }\NormalTok{(z}\OperatorTok{*}\NormalTok{sigma}\OperatorTok{/}\NormalTok{me)}\OperatorTok{^}\DecValTok{2}
\end{Highlighting}
\end{Shaded}

Luke would need a minimum of 665.64 = 666 students.

\subsection{5.20 High School and Beyond, Part
I.}\label{high-school-and-beyond-part-i.}

\begin{enumerate}
\def\labelenumi{(\alph{enumi})}
\tightlist
\item
  Is there a clear difference in the average reading and writing scores?
\item
  Are the reading and writing scores of each student independent of each
  other?
\item
  Create hypotheses appropriate for the following research question: is
  there an evident difference in the average scores of students in the
  reading and writing exam?
\item
  Check the conditions required to complete this test.
\item
  The average observed difference in scores is
  \(\bar{x}_{\text{read}-\text{write}} = -0.545\), and the standard
  deviation of the differences is 8.887 points. Do these data provide
  convincing evidence of a difference between the average scores on the
  two exams?
\item
  What type of error might we have made? Explain what the error means in
  the context of the application.
\item
  Based on the results of this hypothesis test, would you expect a
  confidence interval for the average difference between the reading and
  writing scores to include 0? Explain your reasoning.
\end{enumerate}

\subsubsection{(a)}\label{a-1}

There doesn't appear to be a clear difference. The means are slightly
different in the box plot, but the histogram is nearly normal, with a
center close to 0.

\subsubsection{(b)}\label{b-1}

Because a student's reading abilities are related to their writing
skills, I'd say that the two are not independent, but \textbf{paired}.
If the question is asking between students, then yes, I'd say each
student's abilities are independent.

\subsubsection{(c)}\label{c-1}

\(H_0: \mu_{\text{diff}} = 0.\) There is no difference in the average
scores of students in the reading and writing exam.
\(H_A: \mu_{\text{diff}} \neq 0.\) There \emph{is} a difference in
average scores.

\subsubsection{(d)}\label{d}

Independence: Since the sample size \(n = 200 > 30\), and comprises less
than 10\% of high school students nationwide, we can assume that each
student is independent. Skew: There is no strong skew evident in the
histogram. Because both conditions are satisfied, we can apply the
t-distribution.

\subsubsection{(e)}\label{e}

\begin{Shaded}
\begin{Highlighting}[]
\NormalTok{n <-}\StringTok{ }\DecValTok{200}
\NormalTok{df <-}\StringTok{ }\NormalTok{n}\OperatorTok{-}\DecValTok{1}
\NormalTok{sd <-}\StringTok{ }\FloatTok{8.886}
\NormalTok{avg_diff <-}\StringTok{ }\OperatorTok{-}\FloatTok{0.545}
\NormalTok{se <-}\StringTok{ }\NormalTok{sd }\OperatorTok{/}\StringTok{ }\KeywordTok{sqrt}\NormalTok{(n)}
\NormalTok{tdf <-}\StringTok{ }\NormalTok{(avg_diff }\OperatorTok{-}\StringTok{ }\DecValTok{0}\NormalTok{)}\OperatorTok{/}\NormalTok{(se)}
\NormalTok{p <-}\StringTok{ }\DecValTok{2} \OperatorTok{*}\StringTok{ }\KeywordTok{pt}\NormalTok{(tdf, df)}
\end{Highlighting}
\end{Shaded}

Since \$p =\(0.3867831\) \textgreater{} p = 0.05\$, we can't reject the
null hypothesis, and conclude that there is no difference in the average
scores.

\subsubsection{(f)}\label{f}

(From footnote 16 on page 235): It's possible we didn't detect a
difference, and made a Type 2 Error. If we did make an error, we may
have falsely accepted the null hypothesis.

\subsubsection{(g)}\label{g}

Yes. Since we accepted the null hypothesis, which said that the
difference is 0, then I'd expect 0 to be in a confidence interval.

\subsection{5.32 Fuel efficiency of manual and automatic cars, Part
I.}\label{fuel-efficiency-of-manual-and-automatic-cars-part-i.}

Each year the US Environmental Protection Agency (EPA) releases fuel
economy data on cars manufactured in that year. Below are summary
statistics on fuel efficiency (in miles/gallon) from random samples of
cars with manual and automatic transmissions manufactured in 2012. Do
these data provide strong evidence of a difference between the average
fuel efficiency of cars with manual and automatic transmissions in terms
of their average city mileage? Assume that conditions for inference are
satisfied.

\subsubsection{Solution}\label{solution-1}

\(\bar{x}_{\text{Automatic}} = 16.12, \quad s_{\text{Automatic}} = 3.58\).

\(\bar{x}_{\text{Manual}} = 19.85, \quad s_{\text{Manual}} = 4.51\).\\
\(n = 26\).\\
\(df = n - 1 = 25\).

\(H_0: \mu_{A} = 0.\) There is no difference in mpg between automatic
and manual cars.\\
\(H_A: \mu_{M} \neq 0\). There \emph{is} a difference in mpg.

\(\bar{x}_{\text{diff}} = \bar{x}_{A} - \bar{x}_{M} = 16.12 - 19.85 = -3.73\).

\(SE = \sqrt{\frac{s_{A}^2}{n}+\frac{s_{M}^2}{n}} = \sqrt{\frac{(3.58)^2}{26}+\frac{(4.51)^2}{26}} \approx 1.12927.\)

\begin{Shaded}
\begin{Highlighting}[]
\NormalTok{n <-}\StringTok{ }\DecValTok{26}
\NormalTok{df <-}\StringTok{ }\NormalTok{n}\OperatorTok{-}\DecValTok{1}
\NormalTok{meanA <-}\StringTok{ }\FloatTok{16.12}
\NormalTok{sdA <-}\StringTok{ }\FloatTok{3.58}
\NormalTok{meanM <-}\StringTok{ }\FloatTok{19.85}
\NormalTok{sdM <-}\StringTok{ }\FloatTok{4.51}
\NormalTok{xdiff <-}\StringTok{ }\NormalTok{meanA }\OperatorTok{-}\StringTok{ }\NormalTok{meanM}
\NormalTok{se <-}\StringTok{ }\KeywordTok{sqrt}\NormalTok{(((sdA}\OperatorTok{^}\DecValTok{2}\NormalTok{)}\OperatorTok{/}\NormalTok{n)}\OperatorTok{+}\NormalTok{((sdM)}\OperatorTok{^}\DecValTok{2}\NormalTok{)}\OperatorTok{/}\NormalTok{n)}
\NormalTok{tdf <-}\StringTok{ }\NormalTok{(xdiff }\OperatorTok{-}\StringTok{ }\DecValTok{0}\NormalTok{)}\OperatorTok{/}\NormalTok{(se)}
\NormalTok{p <-}\StringTok{ }\DecValTok{2} \OperatorTok{*}\StringTok{ }\KeywordTok{pt}\NormalTok{(tdf, df)}
\end{Highlighting}
\end{Shaded}

Since \$p =\(0.0028836\) \textless{} p = 0.05\$, we reject the null
hypothesis, and conclude that there \emph{is} a difference in mpg
between automatic and manual cars.

\subsection{5.48 Work hours and
education.}\label{work-hours-and-education.}

The General Social Survey collects data on demographics, education, and
work, among many other characteristics of US residents. Using ANOVA, we
can consider educational attainment levels for all 1,172 respondents at
once. Below are the distributions of hours worked by educational
attainment and relevant summary statistics that will be helpful in
carrying out this analysis. (a) Write hypotheses for evaluating whether
the average number of hours worked varies across the five groups. (b)
Check conditions and describe any assumptions you must make to proceed
with the test. (c) Below is part of the output associated with this
test. Fill in the empty cells. (d) What is the conclusion of the test?

\subsubsection{(a)}\label{a-2}

\(H_0: \mu_1 = \mu_2 = ... = \mu_n\). The mean number of hours worked is
the same across all groups.\\
\(H_A: \mu_1 \neq \mu_2 \neq ... \neq \mu_n\). The mean number of hours
worked is not the same.

\subsubsection{(b)}\label{b-2}

Independence: \(n = 1,172 > 30\), and comprises \(< 10\%\) of the
population, so we can assume they're independent.\\
There doesn't appear to be any strong skew, so we can assume the data is
nearly normal. Mean and standard deviation are similar for the most
part, so we can assume variability across all groups is equal.

\subsubsection{(c)}\label{c-2}

\begin{Shaded}
\begin{Highlighting}[]
\CommentTok{# All formulas taken from page 250 in the textbook}

\NormalTok{k <-}\StringTok{ }\DecValTok{5}   \CommentTok{# Categories}
\NormalTok{df <-}\StringTok{ }\NormalTok{k }\OperatorTok{-}\StringTok{ }\DecValTok{1}   \CommentTok{# degrees of freedom}
\NormalTok{n <-}\StringTok{ }\KeywordTok{c}\NormalTok{(}\DecValTok{121}\NormalTok{, }\DecValTok{546}\NormalTok{, }\DecValTok{97}\NormalTok{, }\DecValTok{253}\NormalTok{, }\DecValTok{155}\NormalTok{)               }\CommentTok{# array of totals for each category}
\NormalTok{tot_n <-}\StringTok{ }\KeywordTok{sum}\NormalTok{(n)}

\NormalTok{df_E <-}\StringTok{ }\NormalTok{tot_n }\OperatorTok{-}\StringTok{ }\NormalTok{k}
\NormalTok{tot_df <-}\StringTok{ }\NormalTok{df }\OperatorTok{+}\StringTok{ }\NormalTok{(tot_n }\OperatorTok{-}\StringTok{ }\NormalTok{k)}

\CommentTok{# MSG = SSG / df_G}

\NormalTok{MSG <-}\StringTok{ }\FloatTok{501.54} \CommentTok{# (given)}
\CommentTok{# SSG = MSG * df_G}
\NormalTok{SSG <-}\StringTok{ }\NormalTok{df }\OperatorTok{*}\StringTok{ }\NormalTok{MSG}
\CommentTok{# SSE = SST - SSG}
\NormalTok{SSE <-}\StringTok{ }\DecValTok{267382}
\NormalTok{SST <-}\StringTok{ }\NormalTok{SSE }\OperatorTok{+}\StringTok{ }\NormalTok{SSG}

\CommentTok{# MSE = SSE / df_E}
\NormalTok{MSE <-}\StringTok{ }\NormalTok{SSE }\OperatorTok{/}\StringTok{ }\NormalTok{df_E}

\CommentTok{# F = MSG / MSE}
\NormalTok{F_stat <-}\StringTok{ }\NormalTok{MSG }\OperatorTok{/}\StringTok{ }\NormalTok{MSE}

\NormalTok{Pr <-}\StringTok{ }\FloatTok{0.0682}

\NormalTok{names <-}\StringTok{ }\KeywordTok{c}\NormalTok{(}\StringTok{"df"}\NormalTok{, }\StringTok{"Sum Sq"}\NormalTok{, }\StringTok{"Mean Sq"}\NormalTok{, }\StringTok{"F Value"}\NormalTok{, }\StringTok{"Pr(>F"}\NormalTok{)}
\NormalTok{row_names <-}\StringTok{ }

\NormalTok{col_df <-}\StringTok{ }\KeywordTok{c}\NormalTok{(df, df_E, tot_df)}
\NormalTok{col_sq <-}\StringTok{ }\KeywordTok{c}\NormalTok{(SSG, SSE, SST)}
\NormalTok{col_msq <-}\StringTok{ }\KeywordTok{c}\NormalTok{(MSG, MSE, }\OtherTok{NA}\NormalTok{)}
\NormalTok{col_F <-}\StringTok{ }\KeywordTok{c}\NormalTok{(F_stat, }\OtherTok{NA}\NormalTok{, }\OtherTok{NA}\NormalTok{)}
\NormalTok{col_Pr <-}\StringTok{ }\KeywordTok{c}\NormalTok{(Pr, }\OtherTok{NA}\NormalTok{, }\OtherTok{NA}\NormalTok{)}
\NormalTok{dataf <-}\StringTok{ }\KeywordTok{data.frame}\NormalTok{(col_df, col_sq, col_msq, col_F, col_Pr)}
\KeywordTok{names}\NormalTok{(dataf) <-}\StringTok{ }\KeywordTok{c}\NormalTok{(}\StringTok{"df"}\NormalTok{, }\StringTok{"Sum Sq"}\NormalTok{, }\StringTok{"Mean Sq"}\NormalTok{, }\StringTok{"F Value"}\NormalTok{, }\StringTok{"Pr(>F)"}\NormalTok{)}
\KeywordTok{rownames}\NormalTok{(dataf) <-}\StringTok{ }\KeywordTok{c}\NormalTok{(}\StringTok{"degree"}\NormalTok{, }\StringTok{"Residuals"}\NormalTok{, }\StringTok{"Total"}\NormalTok{)}
\NormalTok{dataf}
\end{Highlighting}
\end{Shaded}

\begin{verbatim}
##             df    Sum Sq  Mean Sq  F Value Pr(>F)
## degree       4   2006.16 501.5400 2.188992 0.0682
## Residuals 1167 267382.00 229.1191       NA     NA
## Total     1171 269388.16       NA       NA     NA
\end{verbatim}

\subsubsection{(d)}\label{d-1}

Since \(P = 0.0682 > p = 0.05\), we can't reject the null hypothesis,
and conclude that there's no difference in number of hours worked across
demographics.


\end{document}
